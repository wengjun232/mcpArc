\section{Conclusion}\label{sec:conclusion}
ALD-coated MCP-PMTs have the higher CE,
while cause the large charges in SER charge spectrum shown in Fig.~\ref{fig:spe_sreal}.
An explanation based on the Furman secondary emission model and a voltage-division experiment are proposed.
Through innovative design utilizing a dual high-voltage circuit in voltage-division experiment,
the measurement of the gain response of MCP to electrons with different energies is achieved.
Photoelectrons hitting the upper surface of MCP generate multiple true-secondary electrons entering the channels,
which is the reason for the large charges in the SER charge spectrum of MCP-PMT.

By using the Monte Carlo method, the calculation of the SER charge spectrum of MCP-PMT has been achieved.
By performing the chi-square test, the yield of true-secondary electrons can be predicted
which is the first study on the phenomenon of SEE in pulse mode.
After conducting the chi-square test on 9 PMTs, the predicted yield is around 5.98.
Based on the explanation, a Gamma-Tweedie mixture which is a new model for the SER charge spectrum of MCP-PMT has been proposed.

