\section{SER charge spectrum based on Gamma distribution}\label{gammapossion}
Instead of Gaussian containing a small nonphysical tail less than 0,
the charge distribution is parameterized with a Gamma distribution $\varGamma(\alpha, \beta)$ defined by a scale factor $\alpha$ and the rate fator $\beta$ as Eq.~\eqref{eq:gamma}:
\begin{equation}
    \label{eq:gamma}
    \begin{aligned}
        f_\Gamma(x ; \alpha, \beta) = \frac{x^{\alpha-1} e^{-\beta x} \beta^\alpha}{\Gamma(\alpha)} \quad \text { for } x>0 \quad \alpha, \beta>0 \\
    \end{aligned}
\end{equation}
where $\Gamma(\alpha)$ is the Gamma function.
A Gamma distribution uniquely determined by its expectation \(\frac{\alpha}{\beta}\) and variance \(\frac{\alpha}{\beta^2}\)
which can be converted into the mean and the variance of the Gaussian.

Every multiplication of electrons at the dynodes or MCP channels follows a Poisson distribution~\cite{branchandPoisson}.
A series of such multiplications forms a cascade Poisson distribution~\cite{1955Scintillation}.
It is a Galton-Watson branching process~\cite{Bartlett1963TheTO}
and difficult to perform analytical computations.
Breitenberger summaried and indicated that the shape of SER charge spectrum is between the Poisson distribution
and the Gaussian distribution~\cite{1955Scintillation}.
Prescott proposed to utilize a cascade Polya distribution to
characterize the electron multiplication in PMT
when considering the non-uniformity of the dynode surface~\cite{polya}.
Kalousis approximated the Polya distribution as a Gamma distribution to calibrate PMT
and achieved better calibration results than Gaussian model in Eq.~\eqref{eq:sreal}~\cite{2012Calibration,2020A}.
Similar to Eq.~\eqref{eq:sreal}, the SER charge spectrum based on the Gamma distribution is,
\begin{equation}
    \begin{aligned}
        f(Q) & = P_\pi(n_{\mathrm{PE}};\lambda)\bigotimes f_\Gamma(Q;n_{\mathrm{PE}}\alpha, \beta) \\
    \end{aligned}
    \label{eq:Gamma}
\end{equation}
where \(\frac{\alpha}{\beta}=Q_1\) and \(\frac{\alpha}{\beta^2}=\sigma_1^2\).