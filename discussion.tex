\section{Discussion}\label{sec:discussion}
In experiments, it is also possible to consider incorporating the transit time distribution.
In Sec.~\ref{sec:convolution}, compared to electrons directly entering the channels of the MCP,
secondary electrons undergo a process of moving away from the MCP
before being influenced by the electric field and returning to the MCP
which increases the transit time.
By using a testing system with extremely high time resolution,
it may be possible to observe the three types of secondary electrons.
The transit time distribution may exhibit a Gaussian peak caused by back-scattered secondary electrons after the main peak,
while rediffused secondary electrons and true-secondary electrons contribute to a Gaussian peak near the main peak.
Through fitting the charge spectrum and analyzing the transit time distribution,
it becomes possible to predict the yields of the three types of secondary emission more accurately.
Certainly, $\delta_{\mathrm{ts}}$ in this study is actually the product
of the real SEY and the collection efficiency of the MCP for secondary electrons,
and can be understood as the effective true-secondary emission yield.

By the Monte Carlo simulation,
we propose the use of the Gamma-Tweedie mixture to describe the SER charge spectrum of the MCP-PMT.
While the model is designed for the MCP-PMT, it can be extended to the Dynode-PMT as well,
with the proportion $p_0$ being nearly 0
while considering only one photonelectron hitting the first dynode.
The first dynode is regarded as the MCP upper surface,
the other dynodes are collectively considered as the MCP channels.
However, no photoelectron enters the "channels" directly.
In the future, it is possible to develop calibration methods specifically for this type of MCP-PMT
to fully utilize its performance.
