\documentclass{article}
\usepackage[T1]{fontenc}  
\usepackage[scaled=0.82]{beramono}  
\usepackage{microtype} 
\usepackage{amsmath}
\usepackage{amsfonts}
\usepackage{mathtools}
\usepackage{bm}
\usepackage{slashed}
\usepackage{amssymb}
\usepackage{hyperref}
\usepackage{nameref}
\usepackage{siunitx}
% \usepackage[backend=bibtex]{biblatex}
\usepackage[margin=2cm]{geometry}
% \addbibresource{matching-pursuit.bib}

\DeclareMathOperator{\E}{E}

\title{Study on MCP-PMT Mechanism based on secondary electron emission}
\author{Jun,Weng; Zhang, Aiqiang; Xu, Benda; \\ Center for High Energy Physics}
\date{\today}

\begin{document}
\maketitle
\section{Introduction}\label{sec:Introduction}
\section{The long-tail in MCP-PMT charge response }\label{sec:long-tail}
\subsection{Single photoelectron charge response}\label{subsec:statistical}
When the photomultiplier is in operation, the light incident on the photocathode produces photoelectrons via the photoelectric effect
which is a Poisson process. Then photoelectrons can be collected by the MCP under the influence of an electric field, which is a Bernoulli process.
MCP amplifies a single collected  electron, and when the amplification is large enough (usually more than 100 times),
the resulting charge follows a Gaussian distribution $\mathcal{N} (Q_1, \sigma_1^2)$. When multiple electrons are collected at the same time, the resulting charge
distribution is the sum of the charges generated by the amplification of multiple single electrons,
which can still be described by the Gaussian distribution $\mathcal{N} (nQ_1, n\sigma_1^2)$.
The charge response of MCP-PMTS can be discribed as \refeq{eq:sreal}~\cite{1994Absolute}
\begin{equation}
    \begin{aligned}
        Q_{ideal} & = P(n;\mu)\bigotimes G_n(x;nQ_1,\sqrt{n}\sigma_1)                                                                      \\
                  & =\sum_{n = 0}^{\infty}\frac{\mu^n e^{-\mu}}{n!}\frac{1}{\sigma_1\sqrt{2\pi n}}\exp(-\frac{{(x-nQ_1)}^2}{2n\sigma_1^2})
    \end{aligned}
    \label{eq:sreal}
\end{equation}
$\mu$ is the mean number of photoelectrons collected by MCP.\@ $P(n;\mu) = \frac{\mu^n e^{-\mu}}{n!}$ is the probability of $n$ photoelectrons obseved.

When $\mu$ is less than 0.1, the probability of observing two or more photoelectrons is less than one tenth of the probability
of observing one photoelectron, and the PMT charge response is left with two Gaussian peaks as Fig~\ref{fig:spe_sreal}
which is supposed to be SPE spectrum with pedestal peak($Q=0$) and single PE peak($Q=Q_1$)\cite{Xia_2015}.
\begin{figure}[ht]
    \centering
    \includegraphics[height=7cm]{pic/siglepe.pdf}
    \caption{Single photoelectron charge response of Hamamatsu CR365}\label{fig:spe_sreal}
\end{figure}
\subsection{long tail in MCP-PMT charge response}\label{subsec:tail}
In the performance tests to evaluate MCP-PMT used by JNE, we found a long tail in the charge spectrum of MCP-PMTS in the single photoelectron mode as shown in Fig~\ref{fig:tail}
\begin{figure}[ht]
    \centering
    \includegraphics[height=7cm]{pic/longtail.pdf}
    \caption{Single photoelectron charge response of MCP-PMT PM2112$-$9089F}\label{fig:tail}
\end{figure}
\section{Statistical model of MCP-PMT charge response}
\subsection{The structure of MCP-PMT}\label{subsec:structure}
When the photon reaches the photocathode, it is converted into photoelectron through photoelectric conversion
Under the influence of electric field, it reaches the microchannel plate~(MCP). The microchannel plate is coated
with secondary emission material of certain thickness. The electrons interact with the surface of the microchannel plate to generate multiple electrons.
These electrons interact with the surface of the microchannel plate continuously under the action of electric field to form multiple electrons.
Finally, they reach the anode to form electrical signals that can be directly measured.
\begin{figure}[ht]
    \centering
    \includegraphics[height=5cm]{pic/structure.pdf}
    \caption{After entering the MCP channel, the electron collides with the channel wall for many times and multiplies in a cascade process}\label{fig:structure}
\end{figure}

In fact, when photoelectrons reach the upper surface of the MCP,
they may enter the MCP channel directly and undergo a cascade multiplication process,
which is described in Fig~\ref{fig:structure}.
It may also collide with the upper face of the MCP.\@
The photoelectron will not be amplified because of absorption process and so on.
The collection efficiency is then limited to the proportion of the open area of
the channel to the surface area of the MCP.\@

In order to improve the collection efficiency, it is necessary to make more photoelectrons observed.
The MCP upper surface of the MCP-PMTs used by JNE is coated with a layer of secondary emission material\cite{zzj2021Al}
called atomic layer deposition (ALD) technique which is used to improve the lifetime of the MCP-PMT
at first and  was later found to increase the found that the gain, the single electron resolution
and the peak-to-valley ratio of MCP-PMT \cite{2021Effects}.
\begin{figure}[ht]
    \centering
    \includegraphics[height=5cm]{pic/ALD.pdf}
    \caption{$Al_2O_3/MgO/Al_2O_3$}\label{fig:A LD}
\end{figure}

Therefore, the electrons from the upper surface of MCP can also enter the MCP hole for multiplication
under the influence of electric field after the collision between photoelectrons and the upper surface of MCP.\@
This means that MCP-PMT can work in two modes, and each photoelectron can only be amplified in one of these modes.
We define the mode that directly enter the channel for multiplication as \textit{channel mode} and
the mode that enters the channel for multiplication after colliding with the upper surface as \textit{surface mode}.
\begin{figure}[ht]
    \centering
    \includegraphics[height=5cm]{pic/MCPelectron.pdf}
    \caption{The photoelectrons directly enter the channel and enter the channels after hitting the upper face of the MCP}\label{fig:MCP}
\end{figure}
\subsection{double gamma model}\label{subsec:doublegamma}
Considering two multiplication modes, both of which multiply a single electron by a factor of $10^7$,
Their corresponding charge response can be described by Gaussian distribution respectively.
These two modes are mixed together in proportion to form the single photoelectronic response of MCP-PMTs as \refeq{eq:doublegaus}
\begin{equation}
    \label{eq:doublegaus}
    \begin{aligned}
         & Q_{MCP-PMT} = p\times Q_{channel\  mode} + (1-p)\times Q_{surface\  mode} \\
         & Q_{channel\  mode} \sim \mathcal{N} (\mu_1, \sigma_1)                     \\
         & Q_{surface\  mode} \sim \mathcal{N} (\mu_2, \sigma_2)
    \end{aligned}
\end{equation}
Use double Gaussian distribution to fit the charge response:
\begin{figure}[ht]
    \centering
    \includegraphics[height=5cm]{pic/doubleGauss.pdf}
    \caption{The proportion of \textit{channel mode} is bigger than \textit{surface mode}}\label{fig:doubleGauss}
\end{figure}

From~\ref{fig:doubleGauss}, there's still a probability when the charge is zero, which doesn't make physical sense.

why gamma?

use double gamma distribution to fit:
\begin{figure}[ht]
    \centering
    \includegraphics[height=5cm]{pic/doubleGamma.pdf}
    \caption{The proportion of \textit{channel mode} is much less than \textit{surface mode}}\label{fig:doubleGamma}
\end{figure}
Using gamma distribution to fit, the probability when the charge is zero is zero.

\subsection{dark noise from MCP}\label{subsec:dnr}
The large chi-square value indicates that the fitting does not agree well with the SPE measured by experiment. We
found a small peak in $[30,50]$\si[]{ADCns} hypothesized due to dark noise. In MCP-PMT, there are three possible types of noise:
\begin{enumerate}
    \item {Dark noise(DR) mainly caused by the thermionic emission from the
          photocathode\cite{2019wenlj}}
    \item {After pulse caused by the ion feedback from the ionization of
          the residual gas or the amplification process\cite{2019wenlj}}
    \item {MCP noise caused by electron emission from the surface of the MCP}
\end{enumerate}
The first noise creates a charge response similar to that of a single photoelectron with random trigger time
and relatively small trigger rate. The second noise signal is related to the trigger signal of the light source,
the trigger rate of the light source, the greater the After pulse noise trigger rate.

We set the photocathode voltage to the same potential as the upper surface of the MCP, turn off the laser,
and test the dark noise from the MCP itself in the dark environment, get the  charge spectrum as~Fig\ref{fig:MCPDNR}:
\begin{figure}[ht]
    \centering
    \includegraphics[height=5cm]{pic/MCP_DNR.pdf}
    \caption{Trigger ratio = 1.19\%}\label{fig:MCPDNR}
\end{figure}
MCP noise described by Gaussian  is considered in the fitting:
\begin{figure}[ht]
    \centering
    \includegraphics[height=5cm]{pic/gaussDgamma.pdf}
    \caption{Dark noise count rate(DCR) is \SI{3.92}{kHz} and dark noise count rate from MCP(MCP DNR) is \SI{0.245}{kHz}}\label{fig:gausDgam}
\end{figure}
\subsection{secondary electron emission}\label{subsec:fuman}
Photoelectrons are produced with very little energy, about the size of a few electronvolts, and the energy arriving at MCP is determined primarily by the electric field.
When the photoelectron is multiplied in channel mode, the energy entering the channel is the energy accelerated by the electric field, which is about \SI{600}{eV}.
Different from channel mode, the energies of electrons entering the channel is complicated,
and we use Fuman Model to discribe the energy distribution of secondary electrons\cite{2002Probabilistic}.

\subsection{Gain at low incident energy}\label{sec:gain}
From Sec~\ref{subsec:fuman}, the energy of secondary emission electrons is less than the energy of primary electrons and
they are small usually around \SI{100}{eV}.


\section{secondary electron emission affects TT distribution}\label{sec:TT}
\newpage
\bibliographystyle{plain}
\bibliography{ref}

\end{document}